\begin{centering}
	\Large{Resumen\\}\bigskip
\end{centering}
\noindent
Una proteína inmersa en una membrana que se modela de forma continua tiene al menos parte de su superficie encerrada entre lo que se puede considerar como dos interfaces, planas para efectos de modelación. Este trabajo busca modelar este tipo de moléculas para calcular su energía de solvatación, aplicando teoría de cargas imágenes mediante el método de elementos de borde (\texttt{BEM}). Se modifica el programa \texttt{PyGBe}, que ya tiene la capacidad de resolver la ecuación de Laplace con \texttt{BEM} en sistemas proteína - solvente, para incluir el aporte de una membrana definida como el medio entre dos interfaces planas. Para esto se utiliza una función de Green que representa el efecto del solvente por fuera de la membrana de forma matemática, evitando el problema de tener que discretizar los dos planos infinitos correspondientes. La ecuación de Green que se utiliza no considera una distribución iónica en el solvente.\\\\
Los resultados obtenidos se contrastan con un software que se considera estándar actualmente - \texttt{APBSmem}. Se realiza también una comparación en el consumo de memoria y los tiempos de cálculo. Este trabajo es una prueba de concepto, y como tal la comparación con \texttt{APBSmem} es alentadora, pues el error porcentual es bajo y el tiempo de cómputo es normalmente inferior. El principal problema surge para mallas normales en cuanto a refinamiento, en moléculas grandes, pues el uso de memoria se dispara a causa del número de imágenes necesario para obtener una buena precisión. En síntesis, el método de cálculo es funcional, pero tiene falencias en cuanto a eficiencia que lo hacen poco atractivo como herramienta para casos más realistas.