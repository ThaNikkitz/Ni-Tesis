\prefacesection{Agradecimientos}
\textit{Terminé de escribir la memoria, ahora viene la parte más fácil, faltan los puros agradecimientos}. Eso pensé cuando, evidentemente, por fin terminé de ajustar el formato y faltaban solo los agradecimientos. Pues bien, el problema es que no es fácil concentrar todo por cuanto tengo que agradecer, ni mucho menos nombrar a todos aquellos a quienes les debo dicha gratitud, en un par de páginas. Pero hay que intentarlo. Y si, voy a ser cursi.\\\\
Dicen que con tres puntos de apoyo se logra la estabilidad. En mi caso, tienen nombre y son muy especiales porque, independiente de cuánto hayan tenido que hacer o el tipo de día que ellos hubiesen pasado, siempre me brindaron su tiempo para escuchar mis quejas cuando tenía los ánimos por el suelo. Los nombro primero porque no tenían ningún deber conmigo, y sin embargo estuvieron ahí cuando los necesitaba, ya fuera para darme un consejo, hacerme reir, o simplemente decirme que en realidad me estaba amargando por tonteras. Por acompañarme en las buenas y en las malas, gracias, sin ningún orden en particular, a Willy, Santi y Katy por todo su apoyo.\\\\
Gracias a mis profesores, partiendo por Christopher Cooper. Siempre me recibió, por muy ocupado que se encontrara; se hizo el tiempo para explicarme cosas, independiente de lo básicas, avanzadas o intrincadas que resultaran. Gracias por ser un gran profesor guía, confiar en que tenía potencial para aprender, en este par de años, un tema que resulta intimidante y que yo desconocía por completo, y también por reconocer el esfuerzo que tomó. Gracias al profesor Harvey Zambrano, porque aunque hemos conversado pocas veces hasta ahora, cada una de estas charlas fue muy útil, tanto en temas de vocación como de crecimiento personal. Ambos me hablaron desde la experiencia - espero que no se tome como ofensa - y ese tipo de conocimiento no se transmite en las aulas. Probablemente es uno de los tipos de diálogo más fructíferos, pues consiste en compartir cosas que trascienden la técnica, y tristemente uno de los que menos se da entre profesores y alumnos, al menos según mi percepción. Yo tuve mucha suerte en ese sentido, pues mis profesores siempre tuvieron la mejor de las disposiciones para conmigo.\\\\
Gracias al sinnúmero de personas que hicieron que no solo este trabajo, sino también todo el camino que llevó hasta su culminación, fueran posibles. A quienes me ayudaron a resolver problemas de diversas índoles, a quienes me acompañaron a tomar un café, en fin, a quienes compartieron conmigo esta etapa, sin importar desde cuando o por cuanto tiempo.\\\\
Finalmente, mi mayor gratitud va para mi familia, principalmente a mis padres, porque honestamente si yo hubiese sido uno de ellos me habría echado de la casa. Gracias por la paciencia infinita, gracias por perdonarme todo y concederme tanto, gracias por atenderme para hacer todo más fácil y llevadero. Y lamento cada mal momento que les hice pasar, y cada preocupación que tuvieron por mi. No hay palabras que permitan expresarles cuánto aprecio todo lo que han hecho por mi y todo el cariño que les tengo.\\\\
A todos los que han estado en mi vida de una u otra forma, sepan que este logro fue gracias a ustedes y por lo mismo, es tan suyo como mío.
