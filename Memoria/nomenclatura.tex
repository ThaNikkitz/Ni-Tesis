\prefacesection{Nomenclatura}

\begin{tabular}{ll}
	\textbf{Símbolo} & \textbf{Descripción} \\
	& \\
	$a$ & Espesor de membrana   \\
	$\vec{D}$ &  Desplazamiento eléctrico \\
	$\vec{E}$ & Campo eléctrico macroscópico \\
	$\vec{E}^{loc}$ & Campo eléctrico local \\
	$\vec{E}^{ext}$ & Campo eléctrico externo \\
	$\vec{E}^{ind}$ & Campo eléctrico inducido \\
	$e$ & Carga del electrón   \\
	$G_L$ & Función de Green para la ecuación de Laplace \\
	$G_M$ & Función de Green para una carga entre dos interfaces \\
	& planas infinitas paralelas \\
	$G_Y$ & Función de Green para la ecuación PB \\
	$G_t$ & Valor teórico por extrapolación de Richardson \\
	$k_B$ &   Constante de Boltzmann	\\
	$K_L$ & Operador Double-Layer Potential de Laplace  \\
	$K_M$ & Operador Double-Layer Potential en la membrana  \\
	$K_Y$ & Operador Double-Layer Potential de Yukawa  \\
	$N_q$ & Número de cargas físicas reales \\
	$n$ &  Number density, número de cargas imagen    \\
	$n^0$ & Number density medio \\
	$\hat{\mathbf{n}}$ & Vector normal unitario \\
	$\vec{P}$ & Densidad de polarización    \\
\end{tabular}
\newpage
\begin{tabular}{ll}
	$P$ & Tasa de convergencia en extrapolación de Richardson \\
	$\mathbf{p}$ & Momento dipolar     \\
	$q$ & Valor de una carga puntual   \\
	$T$ &   Temperatura  \\
	$\hat{\mathbf{t}}$ & Vector tangencial unitario \\
	$V_L$ & Operador Single-Layer Potential de Laplace  \\
	$V_M$ & Operador Single-Layer Potential en la membrana  \\
	$V_Y$ & Operador Single-Layer Potential de Yukawa  \\
	$z$ & Carga de una especie iónica, tercera coordenada\\
	 & cartesiana \\
\end{tabular}
\\
\\

\begin{tabular}{ll}
	$\alpha$ &  Polarizabilidad microscópica    \\
	$\delta(\mathbf{r})$ & Delta de Dirac     \\
	$\varepsilon$ & Permitividad eléctrica de un medio \\
	$\kappa$ &   Factor de corrección de campo eléctrico, inverso\\
	&	del largo de Debye \\
	$\rho$ & Densidad de carga    \\
	$\phi$ & Potencial electrostático\\
	$\sigma$ & Densidad superficial de carga \\
	$\chi$ & Susceptibilidad eléctrica del medio \\
	$\Omega$ & Dominio (Volumen, 3D) \\	
	$\partial\Omega$ & Dominio (Superficie, 2D) \\
\end{tabular}
