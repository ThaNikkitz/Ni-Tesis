\section*{Abstract}
\noindent
A membrane-embedded protein that is modeled through continuum theory has at least part of its surface surrounded by what can be considered as two planar interfaces. This work presents a way of modelling these type of molecules to calculate their solvation energy by applying image charge theory with the Boundary Element Method (\texttt{BEM}). We modified a basic version of the \texttt{PyGBe} solver, which already has the capability to solve the Laplace equation via \texttt{BEM} for a protein - solvent system, to include the effect of a membrane defined as the medium between two infinite planes. To achieve this, we use a Green's function that takes into account the effect of the solvent outside the membrane in a mathematical fashion through the use of image charges, bypassing the problem of having to mesh the interfaces of the membrane. It is worth noting that said Green's function does not consider ion concentrations in the media.\\\\
The results are then compared with a standard in the field - \texttt{APBSmem}. A comparison in terms of memory consumption and calculation times is also done. This work, as a proof of concept, succeeds, because the difference in results with \texttt{APBSmem} is small and calculation times are usually shorter than those  of the reference program. The main problem arises when using high density meshes, or when working with molecules with a big surface, since the memory consumption grows due to the number of image charges necessary to achieve a good level of precision. Therefore as a conclusion, the calculation method is functional, but its faults in terms of efficiency make it a less than attractive tool for realistic cases in its current state.