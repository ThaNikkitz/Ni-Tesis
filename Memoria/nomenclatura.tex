\prefacesection{Nomenclatura}

\begin{tabular}{ll}
	\textbf{Símbolo} & \textbf{Descripción} \\
	& \\
	$A$ & Área   \\
	$a$ & Espesor de la membrana   \\
	$\textbf{B}$ & Matriz de covarianzas de errores del background \\
	$C$ &  Capacidad calorífica, Coeficiente numérico   \\ 
	$\vec{D}$ &  Desplazamiento eléctrico
	$\vec{E}$ & Campo eléctrico macroscópico \\
	$\vec{E}^{loc}$ & Campo eléctrico local \\
	$\vec{E}^{ext}$ & Campo eléctrico externo \\
	$\vec{E}^{ind}$ & Campo eléctrico inducido \\
	$e$ & Carga del electrón   \\
	$F$ & Forzamiento externo   \\
	$f$ & Factor de Coriolis \\
	$G_L$ & Función de Green para la ecuación de Laplace \\
	$G_Y$ & Función de Green para la ecuación PB \\
	$G_M$ & Función de Green para una carga entre dos \\
	& interfaces planas infinitas paralelas \\
	$\vec g$ & Vector de fuerzas de cuerpo     \\
	$H$ & Fuentes de calor, Operador de obsevación  \\
	$J$ & Función de Costo \\
	$\textbf{K}$ & Matriz peso \\
	$K$ & Coeficiente de difusión turbulento, Energía Cinética   \\
	$k$ &   Energía cinética turbulenta	\\
	$L$ &  Largo de Monin-Obukhov    \\
	$l$ & Escala de longitud característica  \\
	$\ell$ & Escala de longitud \\
	$M$ & Número de Mach     \\
	$m$ & Masa, Factor de mapa    \\
	$N$ &  Number density    \\
	$\hat{\matbf{n}}$ & Vector normal unitario \\
	$\vec{P}$ & Densidad de polarización    \\
	$Pr$ & Número de Prandtl    \\
	$\mathbf{p}$ & Momento dipolar     \\
	$Q$ & Fuentes de masa de agua     \\
	$q$ & Valor de una carga puntual   \\
	$\vec{q}$ &  Vector flujo de calor   \\
	$\textbf{R}$ & Matriz de covarianzas de errores de observaciones \\
	$R$ &   Constante de gases ideales, Número de Richardson   \\
	$Re$ & Número de Reynolds    \\
\end{tabular}
\newpage
\begin{tabular}{ll}
	\textbf{Símbolo} & \textbf{Descripción} \\
	& \\
	$Ri$ & Número de Richardson gradiente  \\
	$\mathcal{S}$ & Tasa de deformación filtrada característica\\
	$\overline{\overline{S}}$ & Tensor tasa de deformación    \\
	$s$ & Desviación Estándar \\
	$T$ &   Temperatura  \\
	$\hat{\mathbf{t}}$ & Vector tangencial unitario \\
	$\vec u$ & Vector velocidad     \\
	$u$ & Componente $x$ de la velocidad \\
	$V$ & Rapidez     \\
	$v$ & Componente $y$ de la velocidad  \\
	$w$ & Componente $z$ de la velocidad  \\
	$\vec{x}$ & Vector de estado \\
	$\vec{y}$ & Vector de observación \\
	& \\
	$\alpha$ &  Polarizabilidad microscópica    \\
	$\beta$ & Coeficiente numérico  \\
	$\gamma$ &  Gradiente de temperatura, Coeficiente de Monin-Obukhov  \\
	$\Delta$ & Tamaño de filtro espacial  \\
	$\delta$ & Delta de Kronecker, Perturbación, Derivada discreta     \\
	$\varepsilon$ & Permitividad eléctrica de un medio \\
	$\eta$ & Escala longitudinal de Kolmogorov, Coordenada de presión \\
	$\theta$ &   Temperatura potencial   \\
	$\kappa$ &   Factor de corrección de campo eléctrico, inverso\\
	&	del largo de Debye
	$\lambda$ & Número de onda, Conductividad térmica  \\	
	$\mu$ & Viscosidad dinámica, Peso de columna de aire    \\
	$\nu$ & Viscosidad cinemática     \\
	$\Pi$ & Producción de energía cinética residual \\
	$\rho$ & Densidad de carga    \\
	$\overline{\overline{\sigma}}$ & Tensor de esfuerzos superficiales   \\
	$\overline{\overline{\tau}}$ & Tensor de esfuerzos viscosos    \\
	$\tau$ &  Periodo  \\
	$\Phi$ & Disipación Viscosa  \\	
	$\phi$ & Potencial electrostático\\
	&  Geopotencial     \\
	$\psi$ & Función de influencia de Monin-Obukhov, Latitud    \\
	$\sigma$ & Densidad superficial de carga \\
	$\xi$ & Susceptibilidad eléctrica del medio \\
	$\Omega$ & Dominio (Volumen, 3D) \\	
	$\partial\Omega$ & Dominio (Superficie, 2D) \\
\end{tabular}
\newpage
\begin{tabular}{ll}
	\textbf{Subíndice} & \textbf{Descripción}\\
	& \\
	$a$ & Análisis \\
	$b$ & Global (\emph{bulk}), background \\
	$c$ & Crítico \\
	$D$ & Arrastre \\
	$d$ & Aire seco \\
	$e$ & Tierra \\
	$E$ & Energía \\	
	$f$ & Flujo turbulento, Filtrado \\
	$h$ & Energía, Hidrostrático, Horizontal     \\
	$i$ & Índice mudo para notación indicial     \\
	$j$ & Índice mudo para notación indicial     \\
	$k$ & Cinético    \\
	$m$ & Momentum  	\\
	$N$ & Relacionado a Brunt–Väisälä \\
	$o$ & Observación \\
	$p$ & Presión constante \\
	$R$ & Reynolds \\	
	$r$ & Roce, Referencia, Residual, Rotación     \\
	$S$ & Smagorinsky \\
	$s$ & Superficie    \\
	$t$ & Turbulento, Verdadero, Superior \\
	$v$ & Volumen constante, Virtual, Vapor, Vertical \\
	$\nu$ & Viscosidad Turbulenta \\
	$0$ & Grandes escalas, Rugosidad\\
	$*$ & Fricción, Variable de escalamiento \\	
\end{tabular}

\bigskip
\begin{tabular}{ll}
	\textbf{Superíndice} & \textbf{Descripción}\\
	 & \\
	$\overline{\,\,\,}$ & Componente media, Filtrada      \\
	$'$ & Componente fluctuante, de Submalla  	\\
	$R$ & Reynolds \\
	$r$ & Reynolds, Residual \\
\end{tabular}