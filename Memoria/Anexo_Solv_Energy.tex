\chapter{Energía de solvatación}\label{Anexo:Solv_Energy}
La energía de solvatación es la energía de un sistema solvente - soluto; se presenta una definición extraída de la referencia \cite{glossary_of_chem_terms}.\\
\textit{Es el cambio en la energía de Gibbs cuando un ion o molécula se transfiere desde el vacío a un solvente. Las contribuciones principales a la energía de solvatación provienen de}:
\begin{itemize}
	\item \textit{La energía por la formación de la cavidad en la que se ubican las especies disueltas de soluto}.
	\item \textit{La energía de orientación de los dipolos}.
	\item \textit{La energía de interacción isotrópica, de origen electrostático y de dispersión}.
	\item \textit{La energía anisotrópica por interacciones específicas, por ejemplo enlaces de hidrógeno}.
\end{itemize}
Para efectos de este trabajo, en vista de que se trata exclusivamente el aporte electrostático en sistemas, la energía de solvatación corresponde únicamente al reordenamiento dipolar y, por ende, al campo de reacción del solvente por la inserción de la molécula.
